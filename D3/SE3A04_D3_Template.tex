\documentclass[]{article}

% Imported Packages
%------------------------------------------------------------------------------
\usepackage{amssymb}
\usepackage{amstext}
\usepackage{amsthm}
\usepackage{amsmath}
\usepackage{enumerate}
\usepackage{fancyhdr}
\usepackage[margin=1in]{geometry}
\usepackage{graphicx}
\usepackage{extarrows}
\usepackage{setspace}
\usepackage{float}
%------------------------------------------------------------------------------

% Header and Footer
%------------------------------------------------------------------------------
\pagestyle{plain}  
\renewcommand\headrulewidth{0.4pt}                                      
\renewcommand\footrulewidth{0.4pt}                                    
%------------------------------------------------------------------------------

% Title Details
%------------------------------------------------------------------------------
\title{Deliverable \#3 Template}
\author{SE 3A04: Software Design III -- Large System Design}
\date{}                               
%------------------------------------------------------------------------------

% Document
%------------------------------------------------------------------------------
\begin{document}

\maketitle	
\noindent{\bf Tutorial Number:} T02\\
{\bf Group Number:} G02 \\
{\bf Group Members:} 
\begin{itemize}
	\item Alvin Qian (Team Leader)
	\item Ryan Kumar
	\item Lucas DeBoer
	\item Varun Pathak
	\item Maya Azar
	\item Moustafa Moustafa
\end{itemize}

\newpage

\section{Introduction}
\label{sec:introduction}
% Begin Section

This document provides further information about the SoleMate system architecture, including state chart diagrams, sequence diagrams, and a detailed class diagram. These materials expand on the foundational concepts introduced in Deliverables 1 and 2, adding detailed insights into how SoleMate’s key components work together.

% End Section
\subsection{Purpose}
\label{sub:purpose}
% Begin SubSection
The document shows how different classes and controllers encapsulated within SoleMate transition between states, interact with one another, and enable the shoe-identification capabilities central to the system.
Additionally, this deliverable serves as a resource for technical and non-technical team members seeking clarity on SoleMate’s design. 
% End SubSection

\subsection{System Description}
\label{sub:system_description}
% Begin SubSection
SoleMate is an intelligent shoe-identification platform that allows users to request footwear information, view past or community-wide identifications, and manage their personal shoe history. This platform combines user-friendly interfaces with expert-driven data to provide reliable, context-based recommendations and identifications.
Building on the concepts introduced in previous deliverables, SoleMate employs multiple controllers to validate user requests, route queries to the appropriate domain experts, and retrieve comprehensive shoe-related insights. 

An extended overview of the system description can be found in Deliverable 2. This document acts as an extension
of Deliverable 2, providing more context in form of state charts, sequence diagrams, and a detailed class
diagram.
% End SubSection

\subsection{Overview}
\label{sub:overview}
% Begin SubSection
The remainder of this document details how SoleMate’s components interact to fulfill user needs:

\begin{itemize}
	\item Section 2 presents state chart diagrams for the key controller classes, illustrating transitions from request submissions through to expert queries and user feedback.
	\item Section 3 provides sequence diagrams that capture step-by-step interactions among system components, showing how user requests flow through the platform.
	\item Section 4 offers a comprehensive class diagram, highlighting the relationships and responsibilities of SoleMate’s major classes.
\end{itemize}

% End SubSection

% End Section

\section{State Charts for Controller Classes}
\label{sec:state_charts_for_controller_classes}
% Begin Section
\begin{figure}[H]
    \centering
    \includegraphics[width=0.9\textwidth]{SMEController.drawio.png}
    \caption{SME Controller state diagram illustrating the identification process flow from request submission to expert querying and user feedback.}
\end{figure}

\begin{figure}[H]
    \centering
    \includegraphics[width=\textwidth]{AccountController.drawio.png}
    \caption{Account Controller state diagram showing login, account creation, validation, and management functionality.}
\end{figure}

\begin{figure}[H]
    \centering
    \includegraphics[width=\textwidth]{AccountShoesController.drawio.png}
    \caption{Account Shoes Controller state diagram showing user interactions for viewing personal shoe history and searching shoes identified by all users.}
\end{figure}

% End Section

\section{Sequence Diagrams}
\label{sec:sequence_diagrams}
% Begin Section

\begin{figure}[H]
    \centering
    \includegraphics[width=0.9\textwidth]{S3/image (1).png}
    \caption{User requests shoe identification}
\end{figure}

\begin{figure}[H]
    \centering
    \includegraphics[width=0.9\textwidth]{S3/image (2).png}
	\caption{User views their past shoe searches}
\end{figure}

\begin{figure}[H]
    \centering
    \includegraphics[width=0.9\textwidth]{S3/image (3).png}
	\caption{User searches for previously identified shoes}
\end{figure}

\begin{figure}[H]
    \centering
    \includegraphics[width=0.9\textwidth]{S3/image (4).png}
	\caption{User audits experts’ answers}
\end{figure}

\begin{figure}[H]
    \centering
    \includegraphics[width=0.9\textwidth]{S3/image (5).png}
	\caption{User creates account}
\end{figure}

\begin{figure}[H]
    \centering
    \includegraphics[width=0.9\textwidth]{S3/image (6).jpeg}
	\caption{User logs in}
\end{figure}

\begin{figure}[H]
    \centering
    \includegraphics[width=0.9\textwidth]{S3/image (7).png}
	\caption{User edits account}
\end{figure}



% End Section
\newpage
\section{Detailed Class Diagram}
\label{sec:detailed_class_diagram}
% Begin Section
\begin{figure}[H]
    \centering
    \includegraphics[width=0.9\textwidth]{fullDiagram.png}
	\caption{Overall detailed class diagram}
\end{figure} 

\begin{figure}[H]
    \centering
    \includegraphics[width=0.9\textwidth]{figure1.png}
	\caption{Detailed class diagram subsection 1}
\end{figure}
Figure 12 is above Figure 13 and left of Figure 14.

\begin{figure}[H]
    \centering
    \includegraphics[width=0.9\textwidth]{figure2.png}
	\caption{Detailed class diagram subsection 2}
\end{figure}
Figure 13 is below Figure 12 and Figure 14.

\begin{figure}[H]
    \centering
    \includegraphics[width=0.9\textwidth]{figure3.png}
	\caption{Detailed class diagram subsection 3}
\end{figure}
Figure 14 is above Figure 13 and right of Figure 12.

% End Section

\appendix
\section{Division of Labour}
\label{sec:division_of_labour}
% Begin Section
Include a Division of Labour sheet which indicates the contributions of each team member. This sheet must be signed by all team members.
% End Section

\newpage
\section*{IMPORTANT NOTES}
\begin{itemize}
	\item You do \underline{NOT} need to provide a text explanation of each diagram; the diagram should speak for itself
	\item Please document any non-standard notations that you may have used
	\begin{itemize}
		\item \emph{Rule of Thumb}: if you feel there is any doubt surrounding the meaning of your notations, document them
	\end{itemize}
	\item Some diagrams may be difficult to fit into one page
	\begin{itemize}
		\item It is OK if the text is small but please ensure that it is readable when printed
		\item If you need to break a diagram onto multiple pages, please adopt a system of doing so and throughly explain how it can be reconnected from one page to the next; if you are unsure about this, please ask me
	\end{itemize}
	\item Please submit the latest version of Deliverable 1 and Deliverable 2 with Deliverable 3
	\begin{itemize}
		\item They do not have to be a freshly printed versions; the latest marked versions are OK
	\end{itemize}
	\item If you do \underline{NOT} have a Division of Labour sheet, your deliverable will \underline{NOT} be marked
\end{itemize}


\end{document}
%------------------------------------------------------------------------------